\documentclass{article}
\usepackage{ctex}
\usepackage[a4paper, total={6in, 8in}]{geometry}
\setlength{\parindent}{0pt}
\usepackage{amsmath,amssymb,amsfonts,color}
\begin{document}
\textcolor{blue}{1.证明:范数$\Vert · \Vert$ 的对偶范数满足范数的定义\\
$\Vert z \Vert_*$\, = \,sup\{$z^T$x:$\Vert x \Vert$ $\leq$ 1\} = sup\{$z^T$x:$\Vert x \Vert$ = 1\}}\\
1.证明:
1)非负性,正定性:$\forall z = 0, sup\{z^T x\} = 0 $ so $\Vert 0 \Vert^* = 0 $
if $z \neq 0, \exists x ,\ that z^T x \neq 0$ so sup\{$z^T$x:$\Vert x \Vert$ $\leq$ 1\} $\geq 0$\\
所以 $\forall x \in \mathbb{R}^n, \Vert x \Vert_* \geq 0 $\\
同时,如果 $\Vert x \Vert_* = 0, x = 0$\\
2)齐次性:我们需要证明$\Vert \alpha z \Vert_* = \Vert \alpha \Vert \Vert z \Vert_*$\\
$\Vert \alpha z \Vert_* = sup\{(\alpha z)^T x:\Vert x \Vert \leq 1\} = sup\{\alpha z^Tx:\Vert x \Vert = 1\}$\\
因为 $\alpha$ 是标量,所以可以将$\alpha$ 提出到括号外,于是有\\
$\Vert \alpha z \Vert_* = sup\{\alpha z^Tx:\Vert x \Vert = 1\} = \alpha sup\{ z^Tx:\Vert x \Vert = 1\} = \alpha \Vert z \Vert_*$\\
齐次性成立\\
3)三角不等式:我们需要证明$\Vert z_1 + z_2 \Vert_* \leq \Vert z_1 \Vert_* \Vert z_2 \Vert_*$\\
$\Vert z_1 + z_2 \Vert_* = sup\{ (z_1+z_2)^Tx:\Vert x \Vert = 1\} =  sup\{ z_1^T x +z_2^T x:\Vert x \Vert = 1\}$
根据上确界的性质,有$sup\{ z_1^T x +z_2^T x:\Vert x \Vert = 1\} \leq sup\{ z_1^T x:\Vert x \Vert = 1\} + sup\{z_2^T x:\Vert x \Vert = 1\} = \Vert z_1 \Vert_* \Vert z_2 \Vert_*$\\
三角不等式证毕\\
综上,对偶范数满足范数的定义\\
\textcolor{blue}{2.证明:$z^T$x $\leq$ $\Vert x \Vert$$\Vert z \Vert$}\\
2.证明:
因为 $\Vert z \Vert_* = sup\{z^Tx:\Vert x \Vert \leq 1\} = sup\{z^Tx:\Vert x \Vert = 1\}$\\
所以 $\Vert z \Vert_* \geq z^T \frac{x}{\Vert x \Vert}$(当z 与 x 同方向时,取"=")\\
对等式两边同时乘 $\Vert x \Vert$ 我们得到: $z^T x \leq  \Vert x \Vert \Vert z \Vert_*$ 
证毕


\textcolor{blue}{3.$f: \mathbb{R}^n \rightarrow \mathbb{R},\, f(x) = log\sum_{i=1}^m exp(a_i^Tx+b_i)$, 其中 $a_1, a_2, \cdots , a_m \in  \mathbb{R}^n $,
$b_1, b_2, \cdots , b_m \in  \mathbb{R} $\\
1). 求函数$f$的梯度 $\nabla f(x)$\\
2). 求函数$f$的二阶导数$\nabla^2 f(x)$}\\
3.解:
1)设 $u =\sum_{i=1}^m exp(a_i^Tx+b_i)$ 则 $f(x) = log u$, $\nabla f(x) = \frac{1}{u} * \nabla u$\\
$\nabla u = \sum_{i=1}^m \nabla exp(a_i^Tx+b_i) = \sum_{i=1}^m exp(a_i^Tx+b_i)a_i$\\
故$\nabla f(x) = \frac{1}{\sum_{i=1}^m exp(a_i^Tx+b_i)} \sum_{i=1}^m exp(a_i^Tx+b_i)a_i$\\
2).$\nabla^2 f(x) = \frac{1}{u} \sum_{i=1}^m \exp(a_i^T x + b_i) a_i a_i^T - \frac{1}{u^2} \left( \sum_{i=1}^m \exp(a_i^T x + b_i) a_i \right) \left( \sum_{i=1}^m \exp(a_i^T x + b_i) a_i \right)^T$\\
where $u =\sum_{i=1}^m exp(a_i^Tx+b_i)$



\textcolor{blue}{4.令$A \in \mathbb{R}^m*n , B \in \mathbb{R}^p*n, b \in  \mathbb{R}^m, d \in \mathbb{R}^p$集合\\
$P = \{x \in \mathbb{R}^n : Ax \leq b 且 Bx = d \}$
是凸集吗?为什么?}\\
4.解:集合$P = \{x \in \mathbb{R}^n : Ax \leq b 且 Bx = d \}$ 是凸集,理由如下:\\
对于 $x_1,x_2$ 满足 $Ax_1 \leq b $ 并且 $Ax_2 \leq b, \forall \lambda \in [0,1],A(\lambda x_1 + (1-\lambda)x_2) = \lambda Ax_1 + (1-\lambda)Ax_2 \leq \lambda b + (1-\lambda)b = b$\\
故满足不等式约束的集合是凸集\\
对于 $x_1,x_2$ 满足 $Bx_1 = d $ 并且 $Bx_2 = d, \forall \lambda \in [0,1], B(\lambda x_1 + (1-\lambda)x_2) = \lambda Bx_1 + (1-\lambda)Bx_2 = \lambda d + (1-\lambda)d = d$\\
故满足等式约束的集合是凸集
综上P是凸集\\

\textcolor{blue}{5.证明:最大值函数$f(x) = max\{x_1,x_2,\cdots ,x_n\}, x = [x_1,x_2, \cdots , x_n] \in \mathbb{R}^n$为凸函数}\\
5.证明:$\forall x,y \in \mathbb{R}^n, \forall \lambda \in [0,1],f(\lambda x+ (1-\lambda)y) = max\{ \lambda x + (1-\lambda)y\}$\\
根据最大值的性质,$ max\{ \lambda x + (1-\lambda)y\} \leq max\{ \lambda x\} + max\{ (1-\lambda)y\} = f(\lambda x) + f((1 - \lambda)y) = \lambda f(x) + (1-\lambda)f(y)$\\
即:$f(\lambda x+ (1-\lambda)y) \leq \lambda f(x) + (1-\lambda)f(y)$\\
故$f(x)$是凸函数


\textcolor{blue}{6.二次函数$f(x) = \frac{1}{2}x^TQx + p^Tx + r,x \in \mathbb{R}^n$是否为凸函数?是否为凹函数?是否为严格凸函数?是否为严格凹函数?}\\
解:$\nabla^2 f(x) = Q$, 由凸函数的二阶判定得:\\如果$Q$为对称半正定矩阵,$f(x)$是凸函数;\\如果$Q$为对称半负定矩阵,$f(x)$是凹函数;\\如果$Q$为对称正定矩阵,$f(x)$是严格凸函数;\\如果$Q$为对称负定矩阵,$f(x)$是严格凹函数;\\

\textcolor{blue}{7.对于凸优化问题\\
$min \, f(x)$\\
$s.t. \, \,x \in X$\\
如果目标函数f是可微的,那么可行解$x^*$是最优解当且仅当 $\forall y \in X$ 有 $\nabla f(x^*)^T(y-x^*) \geq 0$}\\
解:$\Leftarrow$:当$\forall y \in X,\nabla^T f(x^*)(y-x^*)\geq 0$为最优解时,因为$f(x)$是凸函数,凸函数有$\forall x,y \in X, f(y) \geq f(x) + \nabla^T f(x)(y-x)$\\
所以$\forall y \in X, f(y) \geq f(x^*) + \nabla^T f(x^*)(y-x^*) $ \\
即$\forall y \in X, f(y) - f(x^*) \geq \nabla^T f(x^*)(y-x^*) \geq 0$\\
即$\forall y \in X, f(y) \geq f(x^*)$, $x* $为最优解\\
$\Rightarrow$:若$x^*$是最优解,但不满足$\forall y \in X,\nabla f(x^*)^T(y-x^*) \geq 0$即 $\exists y_0 \in X,\nabla f(x^*)^T(y-x^*) < 0$\\
设$z = \lambda x^* + (1-\lambda)y_0, \lambda \in [0,1]$, 当$\lambda = 0,z = y_0, \nabla f(x^*)^T(z-x^*) < 0$,
那么$\exists \lambda \in [0,1) that \, f(z)  = f(\lambda x^* + (1-\lambda)y_0) < f(x^*)$,这与$x^*$是最优解矛盾,所有假设不成立\\
综上,可行解$x^*$是最优解当且仅当 $\forall y \in X$ 有 $\nabla f(x^*)^T(y-x^*) \geq 0$\\




\textcolor{blue}{8.证明$x^* = (1,\frac{1}{2},-1) ^ T$是如下优化问题的最优解$min \frac{1}{2}x^TQx + p^Tx + r$\\
$s.t. -1 \leq x_i \leq 1,i = 1,2,3$
其中,
$$
\begin{gathered}
    $ P= $
    \left[
        \begin{array}{cccc}
            13 & 12 & -2 \\
            12 & 17 & 6 \\
            -2 & 6 & 12
        \end{array} 
    \right]
    $ ,q= $
    \left[
        \begin{array}{cccc}
            -22 \\
            -14.5 \\
            13
        \end{array} 
    \right]
    $ ,r = 1$
\end{gathered}
$$}\\

8.证明:设$f(x) = \frac{1}{2}x^TQx + p^Tx + r,$则 $\nabla^2 f(x) = P $
$$
\begin{gathered}
    $ P= $
    \left[
        \begin{array}{cccc}
            13 & 12 & -2 \\
            12 & 17 & 6 \\
            -2 & 6 & 12
        \end{array} 
    \right]
\end{gathered}
$$
为正定矩阵,故$f(x)$为凸函数且可微,显然不等式限制函数也是凸函数,有$\nabla f(x^*)^T = Px^* + q=$   
$$
\begin{gathered}
    \left[
        \begin{array}{cccc}
            13 & 12 & -2 \\
            12 & 17 & 6 \\
            -2 & 6 & 12
        \end{array} 
    \right]
    \left[
        \begin{array}{cccc}
            1 \\
            \frac{1}{2} \\
            -1
        \end{array} 
    \right]
    $+$
    \left[
        \begin{array}{cccc}
            -22 \\
            -14.5\\
            13
        \end{array} 
    \right]
    $=$
    \left[
        \begin{array}{cccc}
            -1 \\
            0 \\
            2
        \end{array} 
    \right]
\end{gathered}
$$
$y - x^* = $
$$
\begin{gathered}
    \left[
        \begin{array}{cccc}
            y_1+1\\
            y_2 \\
            y_3 -2 
        \end{array} 
    \right]
\end{gathered}
$$
故$\nabla f(x^*)^T(y-x^*) = 3- y_1 + 2y_3 \geq 3 - 1 + 2*1 = 0$\\
由题7可知,当$x^*$ 满足$\,\,\forall y \in X,\nabla^T f(x^*)(y-x^*)\geq 0$时 $x^*$是最优解,故$x^* = (1,\frac{1}{2},-1) ^ T$是优化问题的最优解


\end{document}