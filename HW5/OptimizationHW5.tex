\documentclass{article}
\usepackage{ctex}
\usepackage[a4paper, total={6in, 8in}]{geometry}
\setlength{\parindent}{0pt}
\usepackage{amsmath,amssymb,amsfonts,color}
\usepackage{bm}
\usepackage{graphicx}
\usepackage{float} 
\usepackage{caption} 
\usepackage{subfigure}
\usepackage{algorithm}
\usepackage[document]{ragged2e}
\usepackage{xcolor}
\usepackage{setspace}
\usepackage{listings}
\setstretch{1.8}
\setlength{\parindent}{2em}
\begin{document}
{\centering\section*{优化方法第五章作业}}
\textcolor{blue}{总结不等式约束凸优化问题的求解方法及相互联系}\\
\textcolor{blue}{$min \quad f_0(x)$\\$s.t.f_i(x) \leq 0, i = 1,2,\dots m$\\$Ax = b$}\\
假设问题严格可行(Slater约束品行成立):存在$x\in D$满足$Ax = b,f_i(x)< 0$\\
最优解的充要条件:存在最优解$ x^* \in \mathbb{R}^n$和最优对偶$\lambda ^* \in \mathbb{R}^m,v^* \in \mathbb{R}^p$满足KKT方程:\\
$Ax^* = b, f_i(x^*) \leq 0, \lambda_i^* \geq 0, \lambda _i^* f_i(x^*) = 0$\\
$\nabla f_0(x^*) + \sum_{i=1}^{m}  \lambda _i^* \nabla f_i(x^*) + A^Tv^* = 0$\\
可以将等式不等式约束凸优化问题转化为等式约束凸优化问题:定义示性函数$I_{-}\mathbb{R} \rightarrow \mathbb{R}$\\
$$ I_{-}(i)=\left\{
\begin{aligned}
0 & = & u \leq 0 \\
\infty & = & u > 0 \\
\end{aligned}
\right.
$$
这样问题转化为:\\
$min \quad f_0(x) + \sum\limits_{i = 1}^m I_{-}(f_i(x))$\\$s.t.Ax = b$
由于示性函数不可微,所以这样替代无法使用牛顿方法。\\
定义近似示性函数$\hat{I}_-(u):\mathbb{R} \rightarrow \mathbb{R}, \hat{I}_- = -\frac{1}{t}log(-u)$\\
参数t的值越大,近似示性函数越接近示性函数。\\
对数障碍函数:$\phi:\mathbb{R}^n \rightarrow \mathbb{R}, \phi(x) = - \sum_{i=1}^{m}log(-f_i(x))$\\
$\nabla \phi (x) = \sum_{i=1}^{m} \frac{1}{-f_i(x)}\nabla f_i(x)$\\
$\nabla^2 \phi (x) = \sum_{i=1}^{m} \frac{1}{f_i(x)^2}\nabla f_i(x) \nabla f_i(x)^T + \sum_{i=1}^{m} \frac{1}{-f_i(x)}\nabla^2 f_i(x)$\\
这样问题转化为:$min \quad tf_0(x) + \phi(x)$\\$s.t.Ax = b$\\
对于不同的t求解时,可以用上一个t值对应问题的最优解作为初始点开始迭代。\\
中心路径:对任意$t>0$,近似等式约束优化问题能用牛顿方法求解,且存在唯一解,记作$x^*(t)$\\
中心路径$\{x^*(t):t>0\}$所有中心路径上的点$x^*(t)$满足$Ax^*(t) = b,f_i(x^*t(t) < 0)$\\
存在$v \in \mathbb{R}^p,$使得$t\nabla f_0(x^*(t)) + \nabla \phi(x^*(t)) + A^Tv = t\nabla f_0(x^*(t)) + \sum_{i=1}^{m}\frac{1}{-f_i(x^*(t))}\nabla f_i(x^*(t)) + A^Tv = 0$\\
令$\lambda_i^*(t) = \frac{1}{-tf_i(x^*(t))}, v^*(t) = \frac{\hat{v}}{t},then \quad has:\nabla f_0(x^*(t)) + \sum_{i=1}^{m}\lambda_i^*(t) \nabla f_i(x^*(t)) + A^Tv^*(t) = 0$\\
$\lambda_i^*(t),v^*(t)$是对偶可行解,即$x^*(t) = argmin_x L(x,\lambda_i^*(t),v^*(t)) =argmin_x \nabla f_0(x^*(t)) + \sum_{i=1}^{m}\lambda_i^*(t) \nabla f_i(x^*(t)) +v^*(t)^T(Ax-b) $\\
$x^*(t)$和对偶可行解的对偶间隙为$\frac{m}{t}$\\
\textcolor{red}{中心路径条件是KKT最优性条件的连续变形}\\
障碍方法:顺序求解一系列线性约束的极小问题,由内部迭代得到原问题的可行解,每次所得的可行解作为下一次问题的初始点,逐渐增加t的值,直到达到精度要求\\
\textcolor{red}{等式不等式凸优化问题的障碍方法:\\
给定严格可行初始点$x,t = t^0,\mu >0, \epsilon >0, $令$k = 0$\\
中心点步骤:从x开始,求解$x^*(t) = argmintf_0(x) + \phi(x)\quad s.t. Ax=b$\\
改进:$x = x^*(t)$若$\frac{m}{t} \leq \epsilon$退出,否则$t = \mu t$回到中心点步骤}\\
每一次对偶间隙为$\frac{m}{t}$经过n次中心点步骤后,对偶间隙为$\frac{m}{\mu ^k t^0}$,故最多经过$\lceil \frac{log(\frac{m}{\epsilon t^0})}{log \mu} \rceil$次外层迭代,算法达到$\epsilon$精度要求。\\
故$\mu$的选择很重要,$\mu$较小,外层迭代次数多,内层迭代较少,$\mu$较大,外层迭代较少,内层迭代增多。\\
需要进行阶段一来计算一个严格可行初始点,在确定严格可行初始点之后,使用障碍方法求解\\

阶段一方法:\\
情况1:\\
使用障碍方法求解\\
$x^0 \in domf_0 \cap \dots dom f_m $满足$Ax^0 = b$\\
构造等式不等式限制凸优化问题:\\
$\min\limits_{s,x} s$\\
$s.t. f_i(x) \leq s,Ax = b$\\
$s$为不等式约束的最大不可行值的上界,该问题总是可行的$x^0,s^0 > \max\limits_{i = 1,2,\dots,m}f_i(x^0)$\\
如果$s^* <0$,存在满足$f_i(x) \leq 0$的严格可行解,只要$s<0$障碍方法迭代就可以停止\\
$s^* >0$不存在满足$f_i(x) \leq 0$的严格可行解\\
$s^* =0$且最小值在$x^*h,s^* = 0$处达到,那么不存在严格可行解\\
$s^* =0$且最小值不可达到,则不等式组是不可行的\\
情况2:\\
$x^0 \in domf_0 \cap \dots dom f_m $不满足$Ax^0 = b$\\
用不可行初始点牛顿方法求解初始中心点:\\
$\min \limits_{s,x} t^0f_0(x) - \sum_{1}^{m}log(s - f_i(x))$\\
$s.t.Ax = b,s = 0$\\
初始点选择任意$x\in D,s >  \max\limits_{i = 1,2,\dots,m}f_i(x^0)$\\
情况3:\\
不能在函数公共定义域内确定一点\\
用不可行初始点牛顿方法求解初始中心点:\\
$\min \limits_{s,x} t^0f_0(x+z_0) - \sum_{1}^{m}log(s - f_i(x+z_i))$\\
$s.t.Ax = b,s = 0,z_0 = 0,\dots,z_m = 0$\\
初始点选择任意$x\in \mathbb{R}^n,x + z_i \in dom f_i,s >  \max\limits_{i = 1,2,\dots,m}f_i(x^0)$\\
故采用障碍函数法求解等式不等式约束问题:\\
\textcolor{red}{
初始化步骤:阶段一,采用阶段一方法来解决\\
确定$x$满足$f_i(x) < 0,Ax = b,$设定$t,\mu,\epsilon$
中心点步骤:\\
从x开始,对当前t求接近近似约束优化问题$x^*(t) = argmin tf_0(x) + \phi s.t.Ax = b$\\
停止准则:\\
若$\frac{m}{t} \leq \epsilon$退出,否则$t = \mu t$回到中心点步骤\\
}
原对偶中心残差:\\
$$
\begin{gathered}
    t>0,r_t(x,\lambda,v) = \left[
        \begin{array}{cccc}
            r_{dual} \\
            r_{cent} \\
            r_{pri}
        \end{array} 
    \right]
    $=$
    \left[
        \begin{array}{cccc}
            \nabla f_0(x) + Df(x)^T\lambda + A^T \\
            -diag(\lambda)f(x) - \frac{1}{t}\textbf{1} \\
            Ax - b
        \end{array}
    \right]
    \end{gathered}
$$\\
其中
$$
\begin{gathered}
    f(x) = \left[
        \begin{array}{cccc}
            f_1(x) \\
            \vdots \\
            f_m(x)
        \end{array} 
    \right]
    \quad Df(x) = 
    \left[
        \begin{array}{cccc}
            \nabla f_1(x)^T \\
            \vdots \\
            \nabla f_m(x)^T
        \end{array}
    \right]
    \end{gathered}
$$\\
固定t,从满足$f(x) < 0,\lambda > 0$的点$y = (x,\lambda,v)$开始求解非线性方程$r_t(x,\lambda,v)$的牛顿步进$d_y = (d_x,d_\lambda,d_v)$\\
$$
\begin{gathered}
    r_t(y+d_y) \approx r_t(y) + Dr_t(y)d_y = 0 \Leftrightarrow
    \left[
        \begin{array}{cccc}
            \nabla ^2 f_0(x) + \sum_{i =1}^{m}\lambda_i \nabla^2f_i(x) &Df(x)^T & A^T\\
            -diag(\lambda)Df(x) &-diag(f(x)) & 0\\
            A & 0 & 0
        \end{array} 
    \right]
    \left[
        \begin{array}{cccc}
            d_x \\
            d_\lambda \\
            d_v
        \end{array}
    \right]\\
    = \quad -
    \left[
        \begin{array}{cccc}
            r_{dual} \\
            r_{cent} \\
            r_{pri}
        \end{array}
    \right]
    \end{gathered}
$$\\
在$y^k = (x^k,\lambda ^k,v^k)$收敛到极限值之前,$r_t^k$不一定是可行的,所以不能估计出对偶间隙\\
定义代理对偶间隙:$\hat{\eta}(x,\lambda) = -f(x)^T\lambda$\\
如果x是原可行的,$\lambda,v$是对偶可行点,那么代理对偶间隙就是对偶间隙,有:\\
$f_0(x) - g(\lambda,v) = -f(x)^T\lambda = \hat{\eta}(x,\lambda)$\\
此时对应的$t = \frac{m}{-f(x)^T\lambda} = \frac{m}{\hat{\eta}}$\\
\textcolor{red}{原对偶内点法:\\
初始化步骤:确定$x$满足$f_i(x) < 0$设定$t,\mu,\epsilon$(无等实现制满足要求)\\
重复基本步骤:\\
1. 确定t:$t = \mu \frac{m}{\hat{\eta}(x,\lambda)} = \mu \frac{m}{-f(x)^T\lambda}$\\
2.计算原对偶搜索方向$\nabla y_{pd} = (\nabla x_{pd},\nabla \lambda_{pd},\nabla v_{pd})$\\
3.以减少$\Vert r_t(y+s\nabla t_{pd})\Vert_2$为目标进行直线搜索,确定步长$s>0$,令$y = y + s\nabla y_{pd}$\\
停止准则:$\Vert r_{pri} \Vert_2 < \epsilon_{feas}, \Vert r_{dual} \Vert_2 < \epsilon_{feas},\hat{\eta}(x,\lambda) = -f(x)^T\lambda < \epsilon$}\\
原对偶方法仅有一轮迭代,每次迭代更新原对偶变量,且原对偶迭代的值不需要是可行的,一般更有效,展现超线性收敛性质\\

\end{document}